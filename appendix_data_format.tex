\section{Appendix: HDF5 Baseband Data Format Specification}
An ideal data format to hold our baseband data would be easily interpretable by end users and manipulated with custom Python 3 analysis tools as well as established VLBI correlators like~\difx and~\sfxc. Baseband data produced by the full-array baseband systems on CHIME and its outrigger telescopes are saved to~\hdf~files, which are then processed by offline (and later, real-time) beamformers using CHIME/FRB's \texttt{singlebeam} format, whose data ordering reflects CHIME's FX correlator architecture. We introduce the specification for \texttt{singlebeam} data. The \texttt{singlebeam} files can be accessed through either \texttt{h5py} directly or specialized methods in \bba. It is strongly recommended to use \texttt{baseband\_analysis} to make use of 1) Tools for chunking and parallelization over the frequency axis via \texttt{caput}, 2) the offset encoding
of raw baseband data, and 3) metadata which keep track of sign flips in the complex conjugate convention taken by the beamformer upstream, changing the sign convention when the data are loaded into memory.

For~\hdf~files loaded with either method, a complete \texttt{singlebeam} file should have data and metadata attributes as described below. \textbf{Bolded} refers to features that do not exist or are irrelevant for \texttt{singlebeam} files, but which would be a natural way to extend the data format for the pulsar beam data. %(e.g., \texttt{from baseband\_analysis.core import BBData; data = BBData.from\_file(`file.h5')})


%%%%%%%%%%%%%%%%%%%%%%%%%%
\begin{enumerate}
    \item \texttt{data.index\_map} : a dictionary for users to interpret the axes which exist in the \texttt{BBData} dataset. The \texttt{BBData} dataset holds \texttt{np.ndarrays} of data. Here is a list of axes, and metadata describing them:
    \begin{itemize}
        \item Frequency ($N_{\nu} \leq 1024$): \texttt{data.index\_map[`freq'][`centre']} holds the center frequency of each PFB channel, in MHz. Similarly, \texttt{data.index\_map[`freq'][`id']} Holds the frequency ID of each frequency channel as an integer $k$. The mapping from frequency IDs to frequencies (in MHz) is $\nu_k = 800 - 0.390625k$, for $k = 0\ldots 1023$. Because every channel center and frequency ID is specified, the frequency axis is not assumed to be continuous. 
        \item Array element ($N_e \leq 2048$): \texttt{data.index\_map[`input'][`id']} holds the serial numbers of each antenna used to form the synthesized beam. This axis is no longer present in beamformed baseband data, but the metadata still exist to inform the end user which antennas were combined into a tied-array beam.  
        \item Polarization/Pointing ($N_p$ is even): \texttt{data.index\_map[`beam']} is supposed to hold the information about where the beams are formed. Currently it just holds integers $0,1,...2n-1$, where $n$ is the number of unique sky locations which are beamformed. The pointings and antenna polarization (either `S' or `E') are recorded in \texttt{data['tiedbeam\_locations'][:]}. It is possible to do hundreds of pointings offline in multiple phase center mode (~\citep{leung2021synoptic}), limited only by the size of the \texttt{singlebeam} file produced.
        \item Time ($N_t \sim 10^4$): \texttt{data.index\_map[`time'][`offset\_fpga']} holds the index of every FPGA frame after \texttt{data[`time0'][`fpga\_count']}, such that for a particular element of baseband data in array of shape \texttt{(nfreq, ntime)}, the unix time at which the \texttt{k,m} element was recorded is 
        $$\texttt{data.ctime[`time0`][k] + 2.56e-6 * \texttt{data.index\_map[`time'][`fpga\_offset'][m]}}.$$ 
        Only one record of the \texttt{fpga\_offset} is recorded for all frequency channels, since we do not want to record \texttt{data.index\_map[`time'][`fpga\_offset']} independently for each channel (which would double our data volume). %For acquisition modes involving pulsar gating, 
        \end{itemize}
    
    \item \texttt{data[`tiedbeam\_baseband']} : array of shape ($N_{\nu},N_{p}, N_t$)\\ 
            Holds the actual baseband data in an array of complex numbers. The baseband data is perley2017accurate-calibrated such that the mean of the power obtained by squaring the data is in units of \texttt{Janskys * $f_{good}^2$} where $f_{good}$ is the fraction of antennas that are not flagged. The baseband data have an ambiguous complex conjugate convention. Data that obeys the same complex conjugate convention as raw PFB output from the F-engine also has the attribute \texttt{data[`tiedbeam\_baseband`].attrs[`conjugate\_beamform`] = 1}, whereas data that has the opposite convention (data processed prior to October 2020) lacks this attribute.

    \item \texttt{data[`time0'][`ctime']} : array of shape $(N_{\nu})$ \\
        Holds the absolute start time of each baseband dump as a function of frequency channel. Times are formatted as a UNIX timestamp in seconds (since midnight on January 1 1970 in UTC time). Since the baseband dumps start at a different time in each frequency channel, \texttt{ctime} is recorded as a function of frequency channel, disciplined via a GPS-disciplined crystal oscillator, to the nearest nanosecond. The precision of \texttt{ctime} is $\approx \SI{100}{\ns}$ because it is stored as a \texttt{float64}.

    \item \texttt{data[`time0'][`ctime\_offset']} : array of shape $(N_{\nu})$ \\ 
        For most applications using \texttt{ctime} alone is sufficient. However, since a \texttt{float64} cannot hold UNIX timestamps to nanosecond precision ($\approx$ 19 digits), a second \texttt{float64} holds the last few relevant decimal places of the full UNIX time in seconds. Because of the limitations of a \texttt{float64} it is often the case that \texttt{ctime\_offset} is less than several hundreds of nanoseconds.  \texttt{data[`time0'][`ctime']} and
        \texttt{data[`time0'][`ctime\_offset']} can be easily converted to \texttt{astropy.Time} objects using the \texttt{val2} keyword.

    \item \texttt{data[`time0'][`fpga\_count']} : array of shape $(N_{\nu})$ \\
        Holds the FPGA frame count of each frequency channel, where the zeroth frame is the correlator start time, as an unsigned \texttt{int}. Taken together, \texttt{ctime} and \texttt{ctime\_offset} and \texttt{fpga\_count} can be used to calculate the start time of the dump to within a nanosecond. This calculation can be performed for each frequency channel, and the results should be consistent to $\SI{1e-10}{\second}$. 

    \item \texttt{data[`tiedbeam\_locations'][`ra',`dec', or `pol']} : array of shape $(N_p)$\\
    Holds the sky locations and polarizations used to phase up the array.
    \item \texttt{data[`tiedbeam\_locations'][`X\_400MHz',`Y\_400MHz']} : array of shape $(N_p)$\\
    Holds the sky locations used to phase up the array; present in offline beamformed data only. Translation from horizontal to celestial coordinates is done via the \texttt{beam\_model} package available on Github.
    \item \texttt{data[`centroid']} Holds the position of the telescope's effective centroid, measured from (0,0,0) in local telescope coordinates, in meters,  measured in an Easting/Northing coordinate system, as a function of frequency channel. This is a function of frequency because the telescope's centroid is a sensitivity-weighted average of antenna positions (Post-beamforming). We do not use the frequency-dependent position at present but the capability exists.
    %\item \texttt{data[`telescope'].attrs[`name']} [Not implemented yet] Holds the name of the station (`chime', `pathfinder', `tone', `allenby', or `greenbank', or `hatcreek') 
    %\item \texttt{data[`n2\_gains']} : array of shape ($N_\nu, N_{ant}$)\\
    %Holds the actual $N^2$ gains used to phase up the telescope. Only present in the real-time system.
    %\item \texttt{data[`telescope'].attrs[`phase\_center\_absolute']} [Not implemented yet] Holds the geodetic locations of ``telescope zero'' in the relative coordinate system using an \texttt{astropy.EarthLocation} object (or some encoding thereof). To measure these positions we use a combination of the NGS Coordinate Conversion and Transformation Tool\footnote{\texttt{https://geodesy.noaa.gov/NCAT/}} and geodesic (lat/long/elev) positions and the NAD83 datum. NAD83 uses the GRS80 geoid, which
        differs from the WGS84 datum, which historically uses the GRS80 geoid but was slightly modified. Note that \texttt{astropy} uses WGS84 and not NAD83! For more details on VLBI-precision positioning for CHIME, see~\footnote{\texttt{https://bao.chimenet.ca/doc/documents/1327}}.
\end{enumerate}


\section{HDF5 Visibilities Data Format Specification}
CHIME Outriggers will have a small number of stations collecting full-array baseband dumps and forming multiple synthesized beams. Since each baseline must be correlated and calibrated independently, we store each baseline and each station as its own independent HDF5 group within a \vlbivis container. Each station contains station-related metadata copied from the~\texttt{singlebeam}~data and autocorrelation visibilities up to some maximum lag, while each baseline holds baseline-related (e.g. calibration) metadata and cross-correlation visibilities. For example, processing data from CHIME and TONE would result in two autocorrelation HDF5 groups (\texttt{vis[`chime'],vis[`tone']},), and one cross-correlation HDF5 group (\texttt{vis[`chime-tone']}).

The cross-correlation visibilities, stored in \texttt{\texttt{vis[`chime-tone'][`'vis']}} are packed in \texttt{np.ndarray}s of shape
%    
$$(N_\nu, N_{c}, N_{p}, N_{p},N_{\ell},N_t)$$
%
The axes are as follows:
\begin{enumerate}
    %\item $N_b$ denotes the number of baselines. In CHIME Outriggers, we only consider baselines involving CHIME (no outrigger-outrigger baselines) for now. This simplifies the accounting and computation because one never has to compensate each dataset in $N-1$ different ways. 
    \item $N_\nu$ enumerates the number of frequency channels. Because fringe-finding involves taking Fourier transforms over the frequency axis, this is fixed to 1024 for now, and infilled with zeros where frequency channels are corrupted by e.g. RFI.
    \item $N_{c} \lesssim10$ enumerates the number of correlation phase centers. Usually one or several ($<10$) phase centers will be used per beam, but \texttt{difxcalc} supports up to 250. Currently, we can assign one phase center per synthesized \texttt{singlebeam} pointing, whose beam width is $0.25 \times 0.25$ degrees). %Are there scientific reasons to expand this capability to multiple phase centers per synthesized beam? A tracking beam may have the sensitivity to see sources less than 1 arcminute away, but in full-array baseband dumps, it only makes sense to correlate at the FRB's position.
    \item $N_p \times N_p$ indicates all possible combinations of antenna polarizations. There are two antenna polarizations for each telescope, and they will be labeled ``south'' and ``east'' to denote ``parallel to the cylinder axis'' and ``perpendicular to the cylinder axis'' directions respectively. Since CHIME/FRB Outriggers have co-aligned, dual-polarization antennas, correlating in a linear basis is straightforward and removes the need for polarization calibration.
    \item $N_{\ell} \sim 20$ indicates the number of integer time lags saved (in units of $\SI{2.56}{\us}$). In principle, only a few ($<10$) are needed, but it is not difficult to compute and save roughly 20 integer lags, which also allows for some frequency upchannelization if desired.
    \item $N_{t} \sim 10^{1-4}$ for FRB baseband data enumerates the number of off-pulses correlated in order to estimate the statistical error on the on-pulse visibilities. However, for a 30-second long tracking beam integration with thousands of short pulse windows centered on individual pulsar pulses, $N_{t}$ can approach $\approx 10^4$ for a long pulsar integration.
\end{enumerate}

We also save the following metadata. At the time of cross-correlation, two \texttt{singlebeam} files are compressed into one visibility dataset. In addition to the metadata in both inputted \texttt{singlebeam} files (as described above) we will save...
\begin{enumerate}
    \item  Software metadata -- \texttt{github} commit hash denoting what version of the correlator produced the file.
    \item \texttt{vis[`chime-tone'][`time\_a']} The topocentric start time of each integration (on- and off-pulses) to nanosecond precision (see \texttt{ctime} and \texttt{ctime\_offset} in the previous section), as measured by UNIX time at station ``A'' (the first in the group name, here, CHIME) as a function of frequency channel and time.
    \item \texttt{vis[`chime-tone'][`vis'].attrs[`station\_a',`station\_b']}: \texttt{Astropy.EarthLocation} objects denoting the geocentric (X,Y,Z) positions of the stations fed into \difxcalc.
    \item \texttt{vis[`chime-tone'][`vis'].attrs[`calibrated']} is a boolean attribute denoting whether phase calibration has been applied to the visibilities.
    \item \texttt{vis[`chime-tone'][`vis'].attrs[`clock\_jitter\_corrected',`clock\_drift\_corrected']} Refer to whether one-second timescale clock jitter (between the GPS and maser) has been calibrated out, and weeks-long timescale clock drift (between masers at two stations) has been calibrated out.
    %\item The matched filter used in the integration. This might be a pulse window limit at first (start/stop indices) but could generally be a matched filter made out of the burst autocorrelation pulse profile, from e.g. \texttt{fitburst}.
    %\item The fiducial TOA of the burst at CHIME as a function of frequency, measured at the bottom of the lowest channel in the band. This turns out to be \SI{400.390625}{\mega\hertz} for data channelized into 1024 spectral channels. which was used for integer delay compensation.
    %\item The dispersion measure to which the burst was de-smeared, assuming a dispersion constant of $\mathcal{D} = 1/2.41\times10^{-4}$ $(\mathrm{pc-cm^3})^{-1}$. We realize there is some ambiguity here that allows for a lot of confusion~\citep{kulkarni2020dispersion}, but to maintain consistency with legacy CHIME/FRB data, we stick to this old convention.
    \item \texttt{vis[`chime'][`auto'][`station']} also holds \texttt{Astropy.EarthLocation} objects denoting the geocentric (X,Y,Z) positions of the station.
    \item All metadata stored in the \texttt{BBData} object, e.g. \texttt{bbdata.index\_map} are saved to the \texttt{vis[`chime']} object.
\end{enumerate}