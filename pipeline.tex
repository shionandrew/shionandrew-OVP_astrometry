\section{VLBI Localization Procedure}
Having laid out the various algorithms at our disposal, we construct our VLBI localization pipeline as shown in Fig.~\ref{fig:block_diagram}. Our data processing begins locally at each station, where individual antennas at each station are grouped into one effective single dish with beamforming. The tied-array (or ``single-beam'') data is then correlated at a central site to produce visibilities for VLBI calibration and localization. For more details about the beamforming, we encourage the reader to reference~\citep{masui2019algorithms} for a theoretical treatment and~\citep{michilli2020analysis} for an overview of the CHIME/FRB baseband pipeline.

\begin{enumerate}
    \item \textbf{Form tied-array beams at each station:} We begin (in the upper left circle in Fig.~\ref{fig:block_diagram}) with an initial guess of the position, referred to hereafter as $\nhat_0$. This position needs to be correct at the arcminute level, and can be calculated from a catalog or using the CHIME/FRB baseband pipeline, which can localize single pulses to sub-arcminute precision in the right ascension (R.A.) and declination (Dec.) directions using baseband data from each antenna of CHIME. At each station, we form a tied-array synthesized beam in the direction of $\nhat_0$ to produce ``singlebeam'' baseband data (lower middle of Fig.~\ref{fig:block_diagram}).
    \item \textbf{Form coarse visibilities towards $\nhat_0$:} After the data are chunked (see Sec.~\ref{sec:chunking}), the delays are compensated (Sec.~\ref{sec:delay_comp}). By default, the fractional sample correction is applied, although this option may be omitted. The correlation is be performed for all frequencies, baselines, pointings, and time windows using the basic correlator (Eq.~\ref{eq:naive_corr}). If necessary, coherent dedispersion may be applied to the data via
        $H_\varphi$ (Eq.~\ref{eq:h3}). We discard large lags, except for $\approx 20$ lags around zero delay, which contains the VLBI fringe. This outputs on- and off-pulse visibilities, which are written to an \hdf~file (See Sec.~\ref{sec:corr_algorithms}).
    \item \textbf{Perform delay-only fringe fit:} Use~\coda~to find fringes on the source on a baseline-by-baseline basis, and estimate a total delay for each baseline. Apply delay and rate calibration solutions to the data, whether they are obtained through single-pulse or tracking-beam observations. Neglecting ionospheric shifts and residual Earth rotation over the dispersive sweep, we coarsely localize the burst on each baseline to a single strip on the sky within the baseband localization ellipse (upper middle of Fig.~\ref{fig:block_diagram}) (see Sec.~\ref{sec:coarse_loc}). Refer to this position as $\nhat_1$. 
    \item \textbf{Repeat (2)-(3), re-correlating towards $\nhat_1$:} Once the residual delays are reduced using better correlator pointings, write the visibilities to an \hdf~file for the final fringe fit.
    \item \textbf{Perform ionospheric fringe fit:} After applying delay and rate calibration solutions, simultaneously fit for a non-dispersive and a dispersive time delay in the visibilities on each baseline individually using the likelihood in Eq.~\ref{eq:point_src_estimator}. Combining the likelihood over all baselines gives the final localization. If needed, steps (4) and (5) may be iteratively repeated.
\end{enumerate}

\begin{figure*}
    \centering
    \includegraphics[width=\textwidth]{figs/block_diagram.pdf}
    \caption{A high-level description of the various stages of FRB localization. The solid arrows denote the various stages in our pipeline. First, an initial guess of the FRB's initial position is computed, with sub-arcminute precision, from the CHIME/FRB beamformer~\citep{michilli2020analysis}. This allows for fringes to be found, and a coarse localization within the synthesized beam refines the correlator pointing. The data are re-correlated towards the new pointing, which improves the correlation signal-to-noise.}
    \label{fig:block_diagram}
\end{figure*}

By default, for CHIME/FRB Outriggers, we run the correlator with the following settings. At present, we omit outrigger-outrigger baselines because of their lower sensitivity, and operate the correlator using CHIME as the center of our reference frame and instantaneous baseline delays for delay compensation. We apply the fractional sample correction in all observations and we apply coherent dedispersion to narrow our transient gates. We then apply the naive correlation algorithm defined in Eq.~\ref{eq:naive_corr} and write the visibilities to disk after an initial correlation. We apply clock, delay, and rate corrections to the visibilities after correlation, and after calculating a
refined position using $\mathcal{L}_\tau$, we re-correlate and re-calibrate the visibilities before running a $\mathcal{L}_\varphi$ localization. 

\begin{figure*}
    \centering
    \label{fig:crab_loc}
    \includegraphics[width = 0.98\textwidth]{figs/crab_loc.pdf}
    \caption{An example single-pulse localization of a Crab giant pulse, combining a CHIME baseband localization (i.e. using CHIME's internal baselines) with a VLBI localization using phased CHIME and phased KKO \SI{66}{\kilo\meter} away from CHIME. \textbf{Left:} The plus sign and ellipse denote CHIME baseband localization ($1\sigma$), while the narrow vertical stripes denote the main lobe and two sidelobes of the localization fringe from CHIME-KKO. A one-dimensional slice of the fringe structure is plotted in blue below to give a sense of the complicated substructure within the localization lobes. \textbf{Right:} We combine the localization contours into the final localization contour. The most-probable position is shown with a cross in both plots; we zoom in/recenter our plot coordinates on the true position of the Crab in the right panel.}
\end{figure*}
