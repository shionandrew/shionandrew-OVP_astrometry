%%%%%%%%%%%%%%%%%%%%%%%%%%%%%%%%%%%%%%%%%%%%%%%%%%%%%%%%%%%%%%%%%%%%%%%%%%%%
%% Trim Size : 11in x 8.5in
%% Text Area : 9.6in (include Runningheads) x 7in
%% ws-jai.tex, 26 April 2012
%% Tex file to use with ws-jai.cls written in Latex2E.
%% The content, structure, format and layout of this style file is the
%% property of World Scientific Publishing Co. Pte. Ltd.
%%%%%%%%%%%%%%%%%%%%%%%%%%%%%%%%%%%%%%%%%%%%%%%%%%%%%%%%%%%%%%%%%%%%%%%%%%%%
%%


%\documentclass[draft]{ws-jai}
%\documentclass{ws-jai} For JAI
\documentclass[twocolumn]{aastex631}
%\usepackage[flushleft]{threeparttable}
\usepackage{savesym}
\usepackage{tipa} % for KKO full name rendering
\savesymbol{tablenum}
\usepackage{siunitx}
\restoresymbol{SIX}{tablenum}
\sisetup{
    %binary-units,
    range-phrase=\text{--},
    range-units=single,
    separate-uncertainty=true,
    retain-explicit-plus,
    % exponent-mode=scientific,
    % table-format=1e1,
    % print-zero-exponent=true, 
    % print-unity-mantissa=false
    }
\DeclareSIUnit{\jansky}{Jy}
\DeclareSIUnit{\MSPS}{MSPS}
\DeclareSIUnit{\byte}{B}
\DeclareSIUnit{\tecu}{TECu}
\DeclareSIUnit{\bit}{bit}
\DeclareSIUnit{\sample}{S}
\DeclareSIUnit{\dmunit}{pc~cm^{-3}}
\DeclareSIUnit{\millisec}{ms}
\newcommand{\kkoname}{k'ni\textipa{P}atn k'l$\left._\mathrm{\smile}\right.$stk'masqt}
\newcommand{\rmtec}{\mathrm{TEC}}
%\usepackage[%
%     colorlinks=true,
%     urlcolor=blue,
%     linkcolor=blue,
%     citecolor=blue
     %]{hyperref}

\usepackage{cancel}
%\usepackage{natbib}
%\bibpunct{(}{)}{;}{a}{}{,}
\usepackage{aas_macros}
% \setcitestyle{square,numbers,comma,aysep={+}}

\bibliographystyle{aasjournal}
% \bibliographystyle{plainnat}

\usepackage{amsmath, amssymb,commath}

\newcommand{\corrname}{\texttt{PyFX}}
\newcommand{\calc}{\texttt{calc}}
\newcommand{\vlbivis}{\texttt{VLBIVis}}
\newcommand{\sfxc}{\texttt{SFXC}}
\newcommand{\difx}{\texttt{DiFX}}
\newcommand{\difxcalc}{\texttt{difxcalc}}
\newcommand{\caput}{\texttt{caput}}
\newcommand{\nhat}{\widehat{\textbf{n}}}
\newcommand{\hdf}{\texttt{hdf5}}
\newcommand{\vdif}{\texttt{VDIF}}
\newcommand{\bba}{\texttt{baseband\_analysis}}
\newcommand{\coda}{\texttt{coda}}
\newcommand{\kdm}{K_{\textrm{DM}}}
\DeclareSIUnit{\parsec}{pc}
\newcommand{\chimedatb}{\SI{19.3996}{\millisec}} % dispersive smearing over CHIME band
\newcommand{\todo}[1]{{\color{red}$\blacksquare$~\textsf{[TODO: #1]}}}


\received{\today}
\revised{\today}
\submitjournal{ApJ}
\shorttitle{CHIME/FRB Outriggers VLBI Astrometry}
\shortauthors{Andrew et al.}

\begin{document}
\title{CHIME/FRB Outriggers VLBI Astrometry}
%\input{./plot_scripts/authorlist.tex}
\collaboration{99}{(CHIME/FRB Collaboration)}

\keywords{Radio astronomy(1338), Radio transient sources (2008), 
Radio pulsars (1353), Astronomical instrumentation(799), 
Very long baseline interferometry (1769), high energy astrophysics (739), Radio telescopes (1360)}

\begin{abstract}
In this work, we present an  
\end{abstract}

%1) Initial background,
\section{Introduction}
CHIME/FRB Outriggers is the first of several planned telescopes which will use triggered VLBI to pinpoint fast radio transients with high angular resolution. It represents an application of standard VLBI techniques  a relatively unexplored observational phase space. For instance, the large instantaneous field-of-view of observations implies large target-calibrator angular separations and that direction-dependent astrometric corrections may be necessary. The unique combination of detection telescopes which are efficient at low frequencies ($< 1$ GHz) and high angular resolution VLBI stations ($\gtrsim 1$ M$\lambda$), makes ionospheric effects uniquely important in low frequency astrometry. In addition, the transient nature of FRBs and the high event rate of CHIME ($\sim 100$ VLBI localizations/year) demands a well-calibrated astrometric systematic error model for rapid, surefire, and accurate FRB localization at the highest levels of precision.

Here we use commissioning data from the first year of observations with CHIME/FRB Outriggers to study the astrometric uncertainties as a function of observational conditions under which FRBs may be detected. These include factors like burst correlation signal-to-noise (S/N), local RFI conditions at each station, signal bandwidth of both target and calibrator. We also explore direction dependent effects like position within the primary beam, target-calibrator angular offset, and ionospheric errors.

\section{Observations}
CHIME/FRB Outriggers has a large number of full-array baseband data snapshots -- triggered on single pulse detections of pulsars and FRBs -- which reach a depth of $\sim 500$ mJy over a field of view of $\sim 200$ deg$^2$. This often allows \textbf{2-10} astrometric calibrators from the RFC and VLBA calibrator catalogs~\citep{ma1998international,petrov2021wide} to be observed simultaneously as the FRB within the common field of view~\citep{andrew2024vlbi} of CHIME and its outriggers.

\section{VLBI Localization Pipeline}
We describe the current version of the localization pipeline used by CHIME/FRB Outriggers for full-array baseband observations -- originally described in [PYFX PAPER], now systematized and automated for the complete VLBI array. The basic workflow is as follows.

Upon the successful capture of full-array baseband data at all stations and the interferometric localization at the CHIME core~\citep{michilli2021analysis}, we form station beams at each station -- the CHIME core as well as the outrigger stations -- toward a set of pointings which consist of each calibrator in the field of view, as well as the target.

\subsection{Calibrator fringe finding}
For pointings towards calibrators, the voltage data from each station is delay-compensated and integrated over the time duration of the baseband capture using PyFX CITE to find fringes on as many calibrators as possible. We calculate the spectral kurtosis [CITE] from the full array baseband data in each snapshot and find outliers above $m$ median absolute deviations to reject RFI, where $3 < m < 10$ is optimized to increase the S/N of the calibrator fringes.

\subsection{Burst fringe finding}
For the single target pointing, the voltage data from each station is delay-compensated, coherently dedispersed, and boxcar integrated around the burst profile. To optimize the S/N in the target pointing, this is done in a loop over trials at different time-of-arrivals, DMs, and frequency-independent burst widths. For each of these three trial parameters, fast visibilities are formed on each baseline for the full 400 MHz bandwidth. RFI flagging is performed by looking for spectral kurtosis flagger
We search for fringes within the visibilities by performing an FFT and searching for peaks in the delay cross-correlation function, for the full band, as well as within each 100- and 200-MHz subband. 

trials as a function of bandwidth (masking the low band, masking the mid-band, masking the high-band).

to maximize the S/N as detected in VLBI.


\subsection{Calibration and phase referencing}

\subsection{RFI flagging}
Spectral kurtosis flagging is calculated by PyFX during VLBI correlation in Deller 2014.

\subsection{Flux Calibration}
Establishing a compact flux scale at 600Mhz is nontrivial. Single-dish low frequency ($<$1GHz) fluxes (e.g. WENSS) for sources in our survey will be contaminated by extended emission, while compact flux measurements from the VLBA must be extrapolated downward by approximately an order of magnitude. This poses a challenge even for the LBCS, which derived a fitting formula for the flux of their sources by performing a cross-match to the VLBA calibrator catalogue and using the spectral index at GHz frequencies to extrapolate the flux to 110-190MHz.

Nevertheless, we can estimate a rough flux scale for our data. CHIME and each Outrigger station are stand-alone interferometers that observes a bright continuum source daily to calibrate the amplitude and phase of the baseband data on a station by station basis (see CHIME Overview, Section 2.5 and \cite{kko_adam}, Section 2.6). Because all the stations are non-steerable, an additional correction of each station's beam response towards the direction of the VLBI target must also be applied in order to obtain a correlated flux measurement, as was done with CHIME baseband data in CHIME/FRB Catalog 1 \citep{basecat1}. However, while a primary beam model has been developed and validated with measurements of the Sun for CHIME's main lobe and is estimated to have a band-averaged uncertainty of $\sim 10\%$ \cite{CHIMEFRB_CAT1}, the beam response of GBO has never been directly measured. 

In principle, because each of the Outriggers cylinders are rotated and rolled to share the same field of view, the CHIME beam should be very similar to beam response of each Outrigger station. This is qualitatively confirmed at least for the first Outrigger station KKO at a single declination, where KKO autocorrelation measurements over a single solar transit show an hour-angle dependent beam response closely resembling CHIME's \citep{kko_adam}. Hence, we choose to adopt the CHIME beam model to correct for the GBO beam, with a few caveats:

First–since GBO is separated from CHIME by $>10º$ in latitude, we evaluate the CHIME beam model at the target's local GBO beam coordinates in the N-S direction. Second–we consider the possibility that the GBO beam response is offset in both the E-W and N-S direction relative to what we assume in our model in order to estimate a conservative uncertainty on our reported fluxes. An absolute offset of 0.25$^\circ$ and $10^\circ$ in the E-W and N-S direction, respectively, provides a band-averaged flux error of $\sim$50$\%$ when averaging the beam model over all pointings in CHIME's primary beam. Hence, we stress that the fluxes reported in our catalog are meant to provide an \emph{overall} flux scale for our survey, as the errors on a source-by-source basis can be quite high. We also note that because our survey is conducted on a single baseline, all our flux measurements are constrained to a single axis of the source. 

The beamformed beam can be expressed as 
\begin{equation}
    B^{a}_\nu(t,\hat{n}) = b^{a}_\nu (\hat{n})\big(s_\nu(\hat{n},t)+n^{a}_\nu(t) \big)
\end{equation}
where $b^{a}_\nu$ is the beam response of station $a$ at frequency %$\nu$ towards sky position $\hat{n}$, $s_\nu(\hat{n},t)$ denotes the %signal at time $t$, and $n^{a}_\nu(t)$ denotes the noise. The cross %correlated visibilities then are
\begin{equation}
    |V_{\nu}^{a,b}(\hat{n})| = b^{a}_\nu (\hat{n})
\end{equation}

After applying the estimated beam correction, we measure the correlated flux as the peak amplitude of the cross-correlated visibilities after fourier transforming over frequency. The noise is measured as the standard deviation of the off-delay, which is added in quadrature with the beam error to provide the final reported flux uncertainty. 

While in principle noise will contribute to the uncertainty in a correlated flux measurement, in practice our uncertainties are so overwhelmingly dominated by our beam model that we do not bother computing formal errors on our reported fluxes. 


%We measure the correlated flux density 

Assuming the noise to be uncorrelated with the signal, the true correlated flux of the source $F^{ij}_\nu(\hat{n})$ after correlating the station-calibrated baseband data at stations $i$ and $j$ at position $\hat{n}$ in the sky can be expressed as 
\begin{equation}
    |V^{ij}_\nu|=\sqrt{b^i_\nu(\hat{n})b^j_\nu(\hat{n})} \big(F^{ij}_{\nu}+N_{\nu}\big)
\end{equation}
where $|V^{ij}(\nu)|$ denotes the amplitude of the cross correlated visibilities at frequency $\nu$,  $b^i_\nu(\hat{n})$ denotes the beam response of telsecope $i$ at frequency $\nu$ towards direction $\hat{n}$. 



%\todo{fix, include error on beam (assume cal is 20\% off}} The system equivalent flux density (SEFD) of CHIME and GBO are approximately 60Jy and 250Jy, respectively, at 600Mhz. The CHIME-GBO synthesized beam when coherently combined have an SEFD of $\sim$ 3.8Jy over the entire band, therefore our noise floor after integrating 100ms is approximately $\sim$ 19mJy.  

%Since these are the brightest sources in the sky, and since there are not many sources to statistically establish a flux scale, we quote a ``radiometer-equivalent'' flux using the beam model and the GBO \& CHIME SEFD measurements. This may actually be an underestimate because of an effective increase in the system temperature caused by non-compact flux.


\subsection{Polarimetry}

\subsection{Visibility errors}
Not propagated. Doesn't matter

\section{Astrometric performance}


Delay not angular offset.
1) astrometric accuracy vs S/N: 100 continuum or pulsar sources test localizations for each baseline

2) offset for artificially narrowband (400-600, 400-500, 500-600, 600-700, 700-800, 600-800 MHz) vs offset for the full band. GBO, HCO, KKO 100 test localizations of continuum sources each. 

%3) For repeaters, see how offset averages down 

%3) offset for S/N maximizing calibrator versus offset for nearest calibrator for each baseline. 100 targets for each baseline. 

4) S/N unreferenced vs S/N referenced to the signal maximizing calibrator for each baseline.  100 for each baseline.


%5) NCP-Crab loc or NCP-B0329 loc offset versus impact parameter to the sun for GBO and HCO (test for extreme ionosphere effects).  GBO HCO,  5 in each bin of impact parameter, probably 15-50 sources. Cite KKO paper to explain why ionosphere is not needed

6) Error budgeting procedure (error is based on bandwidth/snr/ionosphere prior). In both cases, show normalized error distribution for pulsars (or calibrators if not enough pulsars) (x 100-200) 


\begin{figure*}[h]
    \centering
    \includegraphics[width=\linewidth]{figs/localization_1.pdf}
    \caption{Plot of localization error as a function of signal to noise}     \label{fig:loc1}
\end{figure*}


\subsection{Signal to noise and bandwidth}
% what CHIME signal to noise is each baseline sensitive to? What flux does that correspond to





\begin{figure*}[h]
    \centering
    \includegraphics[width=\linewidth]{figs/localization_2.pdf}
    \caption{Plot of localization error as a function of signal to noise for bottom half of band, top half, middle half. Compare to naiive theoretical value (no ionosphere) and with ionosphere}   
    \label{fig:locbw}
\end{figure*}


\subsection{Calibrator target separation}

\begin{figure*}[h]
    \centering
    \includegraphics[width=\linewidth]{figs/localization_3.pdf}
    \caption{Residual phase/delay as a function of position on sky and cal-tar separation}   
    \label{fig:locbw}
\end{figure*}
\subsubsection{Beam calibration}
\subsubsection{Ionosphere calibration}

\subsection{Final error budgeting}

\begin{figure*}[h]
    \centering
    \includegraphics[width=\linewidth]{figs/localization_3.pdf}
    \caption{Localization error normalized by uncertainty}   
    \label{fig:locbw}
\end{figure*}

\section{Discussion, Future Work, and Conclusion}

\section{Acknowledments}
%\allacks
\appendix
%\section{Appendix: HDF5 Baseband Data Format Specification}
An ideal data format to hold our baseband data would be easily interpretable by end users and manipulated with custom Python 3 analysis tools as well as established VLBI correlators like~\difx and~\sfxc. Baseband data produced by the full-array baseband systems on CHIME and its outrigger telescopes are saved to~\hdf~files, which are then processed by offline (and later, real-time) beamformers using CHIME/FRB's \texttt{singlebeam} format, whose data ordering reflects CHIME's FX correlator architecture. We introduce the specification for \texttt{singlebeam} data. The \texttt{singlebeam} files can be accessed through either \texttt{h5py} directly or specialized methods in \bba. It is strongly recommended to use \texttt{baseband\_analysis} to make use of 1) Tools for chunking and parallelization over the frequency axis via \texttt{caput}, 2) the offset encoding
of raw baseband data, and 3) metadata which keep track of sign flips in the complex conjugate convention taken by the beamformer upstream, changing the sign convention when the data are loaded into memory.

For~\hdf~files loaded with either method, a complete \texttt{singlebeam} file should have data and metadata attributes as described below. \textbf{Bolded} refers to features that do not exist or are irrelevant for \texttt{singlebeam} files, but which would be a natural way to extend the data format for the pulsar beam data. %(e.g., \texttt{from baseband\_analysis.core import BBData; data = BBData.from\_file(`file.h5')})


%%%%%%%%%%%%%%%%%%%%%%%%%%
\begin{enumerate}
    \item \texttt{data.index\_map} : a dictionary for users to interpret the axes which exist in the \texttt{BBData} dataset. The \texttt{BBData} dataset holds \texttt{np.ndarrays} of data. Here is a list of axes, and metadata describing them:
    \begin{itemize}
        \item Frequency ($N_{\nu} \leq 1024$): \texttt{data.index\_map[`freq'][`centre']} holds the center frequency of each PFB channel, in MHz. Similarly, \texttt{data.index\_map[`freq'][`id']} Holds the frequency ID of each frequency channel as an integer $k$. The mapping from frequency IDs to frequencies (in MHz) is $\nu_k = 800 - 0.390625k$, for $k = 0\ldots 1023$. Because every channel center and frequency ID is specified, the frequency axis is not assumed to be continuous. 
        \item Array element ($N_e \leq 2048$): \texttt{data.index\_map[`input'][`id']} holds the serial numbers of each antenna used to form the synthesized beam. This axis is no longer present in beamformed baseband data, but the metadata still exist to inform the end user which antennas were combined into a tied-array beam.  
        \item Polarization/Pointing ($N_p$ is even): \texttt{data.index\_map[`beam']} is supposed to hold the information about where the beams are formed. Currently it just holds integers $0,1,...2n-1$, where $n$ is the number of unique sky locations which are beamformed. The pointings and antenna polarization (either `S' or `E') are recorded in \texttt{data['tiedbeam\_locations'][:]}. It is possible to do hundreds of pointings offline in multiple phase center mode (~\citep{leung2021synoptic}), limited only by the size of the \texttt{singlebeam} file produced.
        \item Time ($N_t \sim 10^4$): \texttt{data.index\_map[`time'][`offset\_fpga']} holds the index of every FPGA frame after \texttt{data[`time0'][`fpga\_count']}, such that for a particular element of baseband data in array of shape \texttt{(nfreq, ntime)}, the unix time at which the \texttt{k,m} element was recorded is 
        $$\texttt{data.ctime[`time0`][k] + 2.56e-6 * \texttt{data.index\_map[`time'][`fpga\_offset'][m]}}.$$ 
        Only one record of the \texttt{fpga\_offset} is recorded for all frequency channels, since we do not want to record \texttt{data.index\_map[`time'][`fpga\_offset']} independently for each channel (which would double our data volume). %For acquisition modes involving pulsar gating, 
        \end{itemize}
    
    \item \texttt{data[`tiedbeam\_baseband']} : array of shape ($N_{\nu},N_{p}, N_t$)\\ 
            Holds the actual baseband data in an array of complex numbers. The baseband data is perley2017accurate-calibrated such that the mean of the power obtained by squaring the data is in units of \texttt{Janskys * $f_{good}^2$} where $f_{good}$ is the fraction of antennas that are not flagged. The baseband data have an ambiguous complex conjugate convention. Data that obeys the same complex conjugate convention as raw PFB output from the F-engine also has the attribute \texttt{data[`tiedbeam\_baseband`].attrs[`conjugate\_beamform`] = 1}, whereas data that has the opposite convention (data processed prior to October 2020) lacks this attribute.

    \item \texttt{data[`time0'][`ctime']} : array of shape $(N_{\nu})$ \\
        Holds the absolute start time of each baseband dump as a function of frequency channel. Times are formatted as a UNIX timestamp in seconds (since midnight on January 1 1970 in UTC time). Since the baseband dumps start at a different time in each frequency channel, \texttt{ctime} is recorded as a function of frequency channel, disciplined via a GPS-disciplined crystal oscillator, to the nearest nanosecond. The precision of \texttt{ctime} is $\approx \SI{100}{\ns}$ because it is stored as a \texttt{float64}.

    \item \texttt{data[`time0'][`ctime\_offset']} : array of shape $(N_{\nu})$ \\ 
        For most applications using \texttt{ctime} alone is sufficient. However, since a \texttt{float64} cannot hold UNIX timestamps to nanosecond precision ($\approx$ 19 digits), a second \texttt{float64} holds the last few relevant decimal places of the full UNIX time in seconds. Because of the limitations of a \texttt{float64} it is often the case that \texttt{ctime\_offset} is less than several hundreds of nanoseconds.  \texttt{data[`time0'][`ctime']} and
        \texttt{data[`time0'][`ctime\_offset']} can be easily converted to \texttt{astropy.Time} objects using the \texttt{val2} keyword.

    \item \texttt{data[`time0'][`fpga\_count']} : array of shape $(N_{\nu})$ \\
        Holds the FPGA frame count of each frequency channel, where the zeroth frame is the correlator start time, as an unsigned \texttt{int}. Taken together, \texttt{ctime} and \texttt{ctime\_offset} and \texttt{fpga\_count} can be used to calculate the start time of the dump to within a nanosecond. This calculation can be performed for each frequency channel, and the results should be consistent to $\SI{1e-10}{\second}$. 

    \item \texttt{data[`tiedbeam\_locations'][`ra',`dec', or `pol']} : array of shape $(N_p)$\\
    Holds the sky locations and polarizations used to phase up the array.
    \item \texttt{data[`tiedbeam\_locations'][`X\_400MHz',`Y\_400MHz']} : array of shape $(N_p)$\\
    Holds the sky locations used to phase up the array; present in offline beamformed data only. Translation from horizontal to celestial coordinates is done via the \texttt{beam\_model} package available on Github.
    \item \texttt{data[`centroid']} Holds the position of the telescope's effective centroid, measured from (0,0,0) in local telescope coordinates, in meters,  measured in an Easting/Northing coordinate system, as a function of frequency channel. This is a function of frequency because the telescope's centroid is a sensitivity-weighted average of antenna positions (Post-beamforming). We do not use the frequency-dependent position at present but the capability exists.
    %\item \texttt{data[`telescope'].attrs[`name']} [Not implemented yet] Holds the name of the station (`chime', `pathfinder', `tone', `allenby', or `greenbank', or `hatcreek') 
    %\item \texttt{data[`n2\_gains']} : array of shape ($N_\nu, N_{ant}$)\\
    %Holds the actual $N^2$ gains used to phase up the telescope. Only present in the real-time system.
    %\item \texttt{data[`telescope'].attrs[`phase\_center\_absolute']} [Not implemented yet] Holds the geodetic locations of ``telescope zero'' in the relative coordinate system using an \texttt{astropy.EarthLocation} object (or some encoding thereof). To measure these positions we use a combination of the NGS Coordinate Conversion and Transformation Tool\footnote{\texttt{https://geodesy.noaa.gov/NCAT/}} and geodesic (lat/long/elev) positions and the NAD83 datum. NAD83 uses the GRS80 geoid, which
        differs from the WGS84 datum, which historically uses the GRS80 geoid but was slightly modified. Note that \texttt{astropy} uses WGS84 and not NAD83! For more details on VLBI-precision positioning for CHIME, see~\footnote{\texttt{https://bao.chimenet.ca/doc/documents/1327}}.
\end{enumerate}


\section{HDF5 Visibilities Data Format Specification}
CHIME Outriggers will have a small number of stations collecting full-array baseband dumps and forming multiple synthesized beams. Since each baseline must be correlated and calibrated independently, we store each baseline and each station as its own independent HDF5 group within a \vlbivis container. Each station contains station-related metadata copied from the~\texttt{singlebeam}~data and autocorrelation visibilities up to some maximum lag, while each baseline holds baseline-related (e.g. calibration) metadata and cross-correlation visibilities. For example, processing data from CHIME and TONE would result in two autocorrelation HDF5 groups (\texttt{vis[`chime'],vis[`tone']},), and one cross-correlation HDF5 group (\texttt{vis[`chime-tone']}).

The cross-correlation visibilities, stored in \texttt{\texttt{vis[`chime-tone'][`'vis']}} are packed in \texttt{np.ndarray}s of shape
%    
$$(N_\nu, N_{c}, N_{p}, N_{p},N_{\ell},N_t)$$
%
The axes are as follows:
\begin{enumerate}
    %\item $N_b$ denotes the number of baselines. In CHIME Outriggers, we only consider baselines involving CHIME (no outrigger-outrigger baselines) for now. This simplifies the accounting and computation because one never has to compensate each dataset in $N-1$ different ways. 
    \item $N_\nu$ enumerates the number of frequency channels. Because fringe-finding involves taking Fourier transforms over the frequency axis, this is fixed to 1024 for now, and infilled with zeros where frequency channels are corrupted by e.g. RFI.
    \item $N_{c} \lesssim10$ enumerates the number of correlation phase centers. Usually one or several ($<10$) phase centers will be used per beam, but \texttt{difxcalc} supports up to 250. Currently, we can assign one phase center per synthesized \texttt{singlebeam} pointing, whose beam width is $0.25 \times 0.25$ degrees). %Are there scientific reasons to expand this capability to multiple phase centers per synthesized beam? A tracking beam may have the sensitivity to see sources less than 1 arcminute away, but in full-array baseband dumps, it only makes sense to correlate at the FRB's position.
    \item $N_p \times N_p$ indicates all possible combinations of antenna polarizations. There are two antenna polarizations for each telescope, and they will be labeled ``south'' and ``east'' to denote ``parallel to the cylinder axis'' and ``perpendicular to the cylinder axis'' directions respectively. Since CHIME/FRB Outriggers have co-aligned, dual-polarization antennas, correlating in a linear basis is straightforward and removes the need for polarization calibration.
    \item $N_{\ell} \sim 20$ indicates the number of integer time lags saved (in units of $\SI{2.56}{\us}$). In principle, only a few ($<10$) are needed, but it is not difficult to compute and save roughly 20 integer lags, which also allows for some frequency upchannelization if desired.
    \item $N_{t} \sim 10^{1-4}$ for FRB baseband data enumerates the number of off-pulses correlated in order to estimate the statistical error on the on-pulse visibilities. However, for a 30-second long tracking beam integration with thousands of short pulse windows centered on individual pulsar pulses, $N_{t}$ can approach $\approx 10^4$ for a long pulsar integration.
\end{enumerate}

We also save the following metadata. At the time of cross-correlation, two \texttt{singlebeam} files are compressed into one visibility dataset. In addition to the metadata in both inputted \texttt{singlebeam} files (as described above) we will save...
\begin{enumerate}
    \item  Software metadata -- \texttt{github} commit hash denoting what version of the correlator produced the file.
    \item \texttt{vis[`chime-tone'][`time\_a']} The topocentric start time of each integration (on- and off-pulses) to nanosecond precision (see \texttt{ctime} and \texttt{ctime\_offset} in the previous section), as measured by UNIX time at station ``A'' (the first in the group name, here, CHIME) as a function of frequency channel and time.
    \item \texttt{vis[`chime-tone'][`vis'].attrs[`station\_a',`station\_b']}: \texttt{Astropy.EarthLocation} objects denoting the geocentric (X,Y,Z) positions of the stations fed into \difxcalc.
    \item \texttt{vis[`chime-tone'][`vis'].attrs[`calibrated']} is a boolean attribute denoting whether phase calibration has been applied to the visibilities.
    \item \texttt{vis[`chime-tone'][`vis'].attrs[`clock\_jitter\_corrected',`clock\_drift\_corrected']} Refer to whether one-second timescale clock jitter (between the GPS and maser) has been calibrated out, and weeks-long timescale clock drift (between masers at two stations) has been calibrated out.
    %\item The matched filter used in the integration. This might be a pulse window limit at first (start/stop indices) but could generally be a matched filter made out of the burst autocorrelation pulse profile, from e.g. \texttt{fitburst}.
    %\item The fiducial TOA of the burst at CHIME as a function of frequency, measured at the bottom of the lowest channel in the band. This turns out to be \SI{400.390625}{\mega\hertz} for data channelized into 1024 spectral channels. which was used for integer delay compensation.
    %\item The dispersion measure to which the burst was de-smeared, assuming a dispersion constant of $\mathcal{D} = 1/2.41\times10^{-4}$ $(\mathrm{pc-cm^3})^{-1}$. We realize there is some ambiguity here that allows for a lot of confusion~\citep{kulkarni2020dispersion}, but to maintain consistency with legacy CHIME/FRB data, we stick to this old convention.
    \item \texttt{vis[`chime'][`auto'][`station']} also holds \texttt{Astropy.EarthLocation} objects denoting the geocentric (X,Y,Z) positions of the station.
    \item All metadata stored in the \texttt{BBData} object, e.g. \texttt{bbdata.index\_map} are saved to the \texttt{vis[`chime']} object.
\end{enumerate}
\onecolumngrid


\bibliography{references}{}

\bibliographystyle{aasjournal}
\end{document}
